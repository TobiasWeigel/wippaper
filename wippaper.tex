% \documentclass[12pt,twocolumn]{article}
% Copernicus stuff
\documentclass[gmd,manuscript]{copernicus}

% page/line labeling and referencing
% from http://goo.gl/HvS9BK
% \newcommand{\pllabel}[1]{\label{p-#1}\linelabel{l-#1}}
% \newcommand{\plref}[1]{page~\pageref{p-#1}, line~\lineref{l-#1}}
% answer environment for reviewer responses
% \newenvironment{answer}{\color{blue}}{}

\hypersetup{colorlinks=true,urlcolor=blue,citecolor=red}
% \newcommand{\degree}{\ensuremath{^\circ}}
% \newcommand{\order}{\ensuremath{\mathcal{O}}}
\newcommand{\bibref}[1] { \cite{ref:#1}}
\newcommand{\pipref}[1] {\citep{ref:#1}}
% \newcommand{\ceqref}[1] {\mbox{CodeBlock \ref{code:#1}}}
% \newcommand{\charef}[1] {\mbox{Chapter \ref{cha:#1}}}
% \newcommand{\eqnref}[1] {\mbox{Eq.     \ref{eq:#1}}}
\newcommand{\figref}[1] {\mbox{Figure   \ref{fig:#1}}}
\newcommand{\secref}[1] {\mbox{Section  \ref{sec:#1}}}
\newcommand{\appref}[1] {\mbox{Appendix \ref{sec:#1}}}
% \newcommand{\tabref}[1] {\mbox{Table   \ref{tab:#1}}}

\runningtitle{WIP Paper Draft \today}
\runningauthor{Balaji et al.}

\begin{document}

\title{Requirements for a global data infrastructure in support of CMIP6}
% \pllabel{SC1-1}

\Author[1,2]{V.}{Balaji}
\Author[3]{Karl}{Taylor}
\Author[4]{Martin}{Juckes}
\Author[5]{Michael}{Lautenschlager}
\Author[6]{Luca}{Cinquini}
\Author[7]{Cecelia}{DeLuca}
\Author[8]{S\'ebastien}{Denvil}
\Author[9]{Mark}{Elkington}
\Author[8]{Francesca}{Guglielmo}
\Author[8]{Eric}{Guilyardi}
\Author[10]{Slava}{Kharin}
\Author[11,4]{Bryan}{Lawrence}
\Author[1,2]{Sergey}{Nikonov}
\Author[12,2]{Aparna}{Radhakrishnan}
\Author[3]{Dean}{Williams}


\affil[1]{Princeton University, Cooperative Institute of Climate
  Science, Princeton NJ, USA}
\affil[2]{NOAA/Geophysical Fluid Dynamics Laboratory, Princeton NJ,
  USA}
\affil[3]{PCMDI}
\affil[4]{Science and Technology Facilities Council, Abingdon, UK}
\affil[5]{Deutsches KlimaRechenZentrum GmbH, Hamburg, Germany}
\affil[6]{NASA/JPL}
\affil[7]{NOAA/ESRL}
\affil[8]{Institut Pierre-Simon Laplace, CNRS/UPMC, Paris, France}
\affil[9]{UKMO}
\affil[10]{CCCma}
\affil[11]{National Centre for Atmospheric Science and University of
  Reading, UK}
\affil[12]{Engility Inc., Dover NJ, USA}
% \affil[10]{NCAR}

\correspondence{V. Balaji (\texttt{balaji@princeton.edu})}

\received{}
\pubdiscuss{} %% only important for two-stage journals
\revised{}
\accepted{}
\published{}

%% These dates will be inserted by Copernicus Publications during the typesetting process.


\firstpage{1}

\maketitle

% \pagebreak
\abstract{ The WGCM Infrastructure Panel (WIP) was formed in 2014 in
  response to the explosive growth in size and complexity of CMIP
  between the CMIP3 and CMIP5 experiments. This article is an account
  of the findings of the WIP on the nature of the global data
  infrastructure needed to support its design and future growth and
  evolution. Based upon these findings, the WIP makes here a set of
  recommendations, in close coordination with the builders of
  infrastructure (e.g the Earth System Grid Federation). The
  recommendations are based on several principles: separation of
  considerations into requirements, implementation, and operations, of
  which this articles focuses on the former; the consideration of data
  as a commodity within a very diverse ecosystem of users: the
  modeling community, the analyst community, the software community,
  and the downstream users of data; the recording and tracking of data
  provenance for scientific reproducibility and accountability; the
  recording and tracking of data usage for the purposes of assigning
  credit; the need for automation in many parts of the process as
  specifications grow ever more complex; methods of measuring the
  costs and benefits of the data. One key element in the design is to
  make most elements dataset-centric rather than system-centric, thus
  making the infrastructure less prone to systemic failure.

  Based upon these principles, the WIP has made a series of
  recommendations in position papers, which are summarized here. The
  recommendations cover the data request itself, an estimate of future
  data volumes and considerations of replication and versioning; a
  system of assigning persistent identifiers for dataset tracking;
  data quality assurance; citation and long-term archival.

  To conclude, we consider aspects of the future evolution of the
  global data infrastructure that follow from the blurring of
  boundaries between climate and weather, as well as the changing
  nature of published scientific results in the digital age.

}
% \pagebreak

\introduction
\label{sec:intro}

CMIP6 \pipref{eyringetal2016a}, the latest incarnation of the
Coupled Model Intercomparison Project, can trace its genealogy back to
the Charney Report \pipref{charneyetal1979}. That seminal report on
the links between CO$_2$ and climate, produced findings that have
stood the test of time \pipref{bonyetal2013}, and it is often noted
that its uncertainty bounds on equilibrium climate sensitivity have
not fundamentally changed, despite the enormous increase in resources
devoted to the problem in the decades since.

Beyond its prescient findings on climate sensitivity, the Charney
Report also gave rise to a methodology for the treatment of
uncertainties and gaps in understanding, which have been equally
influential, and in fact is the basis of CMIP itself. The Report can
be seen as the first use of the \emph{multi-model ensemble}. At the
time, there were two models capable of representing the equilibrium
response of the climate system to a change in CO$_2$ forcing, one from
Syukuro Manabe's group at NOAA's Geophysical Fluid Dynamics
Laboratory, and the other from James Hansen's group at NASA's Goddard
Institute for Space Studies. Then as now, these groups marshaled vast
state-of-the-art computing and data resources to run very challenging
simulations of the Earth system. The Report's results were based an
ensemble of 3 runs from Manabe, labeled M1-M3, and two from Hansen,
labeled H1-H2.

By the time of the IPCC First Assessment Report (FAR) in 1990, the
process had been formalized. By then, there were 5 models
participating in the exercise, and some of the runs that are part of
the DECK today (a pre-industrial control, 1\% per year CO$_2$ increase
to doubling, etc) had been standardized. The ``scenarios'' had emerged
as well, for a total of 5 different experimental protocols. To
fast-forward to today, CMIP6 expects around 65 models from 34 modeling
centers \citep[in 14 countries, in stark contrast to the US monopoly
in][]{ref:charneyetal1979} to participate in the DECK and historical
experiments \citep[Table~2 of][]{ref:eyringetal2016a}, and some subset
of these to participate in one or more the 21 MIPs endorsed by the
CMIP Panel \citep[Table~3 of][]{ref:eyringetal2016a}. The MIPs
themselves contain a large hierarchy of experiments, numbering in the
hundreds.

The complexity of the data request has also grown tremendously over
this period. A typical dataset from the FAR archive
(\href{https://goo.gl/M1WSJy}{from GFDL's R15 model}) lists
climatologies and time series of two variables, and the dataset size
is about 200~MB. The CMIP6 Data Request \citep[][replace with GMD
ref?]{ref:juckesetal2015} lists literally thousands of variables from
the hundreds of experiments mentioned above. This growth in complexity
is testament to our understanding of many physical, chemical and
biological processes which were simply absent from the Charney Report
era models, but we are increasingly confident in our ability to
simulate.

The simulation output itself is now a primary scientific resource for
researchers the world over, rivaling the volume of global
observational weather and climate data from the global array of
sensors and satellites \pipref{overpecketal2011}. Climate science,
and climate simulation in particular, has now become one of the
primary elements in the ``vast machine'' \pipref{edwards2010}
that is used to run the global climate and weather enterprise. The
global map of sites participating in the federated archive known as
the Earth System Grid Federation \pipref{williamsetal2015} is shown
in \figref{esgf}. The federation, resourced as it is by multiple
uncoordinated national and international agencies, is itself a complex
and delicate artifact to manage. As shown in the figure, the different
national nodes may be answerable to multiple projects beyond CMIP5,
with their own, uncoordinated and possibly conflicting, requirements.


\begin{figure*}
  \begin{center}
    \includegraphics[width=175mm]{images/esgf-map-2017.png}
  \end{center}
  \caption{Sites participating in the Earth System Grid Federation in
    2017. Figure courtesy Dean Williams, adapted from the ESGF Brochure.}
  \label{fig:esgf}
\end{figure*}



The sheer size and complexity of this infrastructure emerged as a
matter for great concern at the end of CMIP5, when the growth in data
volume relative to CMIP3 (from 40~TB to 2~PB, a 50-fold increase in 6
years) suggested the community was on an unsustainable path. These
concerns led the WGCM in 2014 to recommend the formation of an
\emph{infrastructure panel}, based upon \href{https://goo.gl/FHqbNN}{a
  proposal at the 2013 annual meeting}. The WGCM Infrastructure Panel
(WIP) was tasked with examining the global computational and data
infrastructure underpinning CMIP, and communicating between the teams
overseeing the scientific and experimental design of these globally
coordinated experiments, and the teams providing resources and
designing that infrastructure. The communication was intended to be
two-way: providing input both to the provisioning of infrastructure
appropriate to the experimental design, and informing the scientific
design of the technical (and financial) limits of that infrastructure.

This paper is a summary of findings from the WIP at the end of 3 years
of activity from its formation in 2014. In \secref{principles}, we
outline the principles and scientific rationale underlying the
requirements for global data infrastructure articulated by WIP. In
\secref{dreq} we cover the CMIP6 Data Request: standards and
conventions, requirements for modeling centres to process a complex
data request, and projections of data volume in CMIP6, all based on
trends in modeling. In \secref{licensing}, we cover recent evolution
in how data are archived, and provide a licensing strategy consistent
with current practice and scientific principle. In \secref{cite} we
discuss issues surrounding data as a first-class resource, including
the technical infrastructure for the creation of citable data, and the
documentation and other standards required to make data a first-class
scientific entity. In \secref{replica} we discuss the implications of
data replicas, and in \secref{version}, issues surrounding data
versioning, retraction and errata. We conclude in \secref{summary}
with an outlook for the future of global data infrastructure, looking
beyond CMIP6, and towards a unified view of the ``vast machine'' for
weather and climate computation and data.

\section{Principles underlying the WIP requirements}
\label{sec:principles}

In the pioneering days of CMIP, the community of participants was
small and well-knit, and all the issues involved in generating
datasets for common analysis from different modeling groups could be
settled by mutual agreement (Ron Stouffer, personal communication).
Analysis was performed by the same community that performed the
simulations. The Program for Climate Model Diagnostics and
Intercomparison (PCMDI) was established in 1989, and gradually took
over the coordination of this process in close cooperation with the
community of modelers. Until CMIP3, the hosting of datasets from
different modeling groups could be managed at a single archival site.
PCMDI alone hosted the entire 40~TB archive.

That juncture can be thought of as a ``tipping point'' in the nature
of the global data infrastructure. CMIP3, and the associated IPCC
assessment report AR4, were a landmark event, honoured with the Nobel
Peace Prize in 2007. As noted above, there subequently was explosive
growth in the scope of CMIP between CMIP3 and CMIP5. That led to
several changes with consequences for the underlying infrastructure.
The WIP's examination of these changes, and the requirements arising
from those changes, are summarized here:

\begin{enumerate}
\item The first of these was the recognition that the global
  computational and data infrastructure itself needed to be formally
  examined as an element in the design of globally coordinated
  scientific experiments. The WIP was formed in response to this
  understanding, and its membership is drawn from players in various
  aspects of the infrastructure. Representatives of modeling centres,
  infrastructure developers, and stakeholders in the scientific
  content and output are all members. One of the WIP's first acts was
  to separate the process of infrastructure development into three
  phases: \emph{requirements}, \emph{implementation}, and
  \emph{operations}. The WIP itself, in consort with the CMIP Panel,
  takes responsibility for articulating the requirements for the
  infrastructure. The implementation is in the hands of the developers
  of infrastructure, principally ESGF for the federated archive
  \pipref{williamsetal2015}, but also related projects like Earth
  System Documentation
  \citep[\href{https://goo.gl/WNwKD9}{ES-DOC},][]{ref:guilyardietal2013},
  as well as the builders of workflows at the modeling centres
  themselves. In addition, we now have the operational aspects of the
  federation itself, as the complete data is now a distributed
  resource. At the WIP's request, the CMIP6 Data Node Operations Team
  (CDNOT) was formed in 2016. Led by WIP member S\'ebastien Denvil, it
  is in charge of making sure that all the infrastructure elements
  necessary for the proper conduct of the CMIP6 experiment are not
  only part of the software as designed, but are actually deployed and
  demonstrably working at the sites charges with hosting CMIP6 data.

  While the WIP separates infrastructure tasks into the categories of
  requirements, implementation, and operations, close links are
  maintained by cross-membership between the key bodies, including the
  WIP itself, the CMIP Panel, the ESGF Executive Committee, and the
  CDNOT.
\item\label{broad} With the basic fact of anthropogenic climate change
  established beyond reasonable doubt, a substantial body of work
  began to move to the area of climate impacts. This meant that the
  centre of gravity of the community of data analysts was no longer
  the specialists in Earth system science -- who also designed and ran
  the experiments and produced the model output -- but data
  \emph{consumers} from allied fields, studying the impacts of climate
  change on health, agriculture, natural resources, human migration,
  and similar issues \pipref{mossetal2010}. This is the problem of
  \emph{scientific scalability}: the data during its lifetime will be
  consumed by a community much larger (both in sheer numbers, as well
  as in breadth of interest and perspective) than the Earth
  system modeling community itself. The WIP therefore resolves
  to build an infrastructure that ensures maximum transparency and
  usability for user (consumer) communities at some remove from the
  modeling (producer) communities.
\item\label{repro} While CMIP and IPCC are formally independent, the
  CMIP archive was recognized as an essential contributor to climate
  policy. This led the IPCC to establish data distribution centres
  (DDCs) that would guarantee the archival and dissemination of this
  data in perpetuity. The \emph{provenance} of data became a matter of
  central importance: being able to trace, long after the fact, back
  from model output to the configuration of models and analysis
  procedures and choices made along the way. Provenance is also an
  important element in achieving the goal of \emph{scientific
    reproducibility}. While in the long-term the scientific method is
  ``self-correcting'', it is in the short-term a source for contention
  and uncertainty \pipref{collinstabak2014}. In a setting like CMIP,
  where a multi-model ensemble is used for both consensus estimates
  and uncertainty bounds on climate projections, it is important to
  document as precisely as possible the details and differences among
  model configurations and analysis methods.
\item\label{analysis} Typical multi-model analysis has hitherto taken
  the multi-model average, equally weighted among models, as a measure
  of consensus. This ``model democracy'' \pipref{knutti2010} has been
  called into question and there is now a considerable literature on
  weighting models by ``quality'' \pipref{knuttietal2017}. This has
  led to community efforts to develop standard metrics of model
  ``quality'' \citep{ref:eyringetal2016,ref:gleckleretal2016}. To
  prevent modeling centres from ``gaming'' the system by tuning for
  scores on specific metrics, the WIP recognizes the need to build an
  infrastructure to support a growing and diverse suite of metrics.
\item As the experimental design of CMIP grew in complexity, the costs
  became a matter of great concern, particularly at the modeling
  centres. While the data were made freely available, the centres, and
  the individual scientists involved in model development, deserve
  fair credit for their contribution. The WIP noted that the current
  citation practices did not lead to recognition of the individual
  modeling groups' contributions, instead most often a single
  collective citation for all of CMIP. We resolved to define a
  mechanism for the creation of citable data, and encouraging its
  practice. In addition to the costs of model development, which must
  be paid back in the currency of scientific recognition, the WIP
  recognized that simply running the suite of CMIP experiments was
  itself a significant burden on the centres. The CPMIP Project
  \pipref{balajietal2017} is an attempt to document these costs in
  CMIP6. Similarly, as archive sizes grow, the WIP recognizes the
  importance of \emph{tracking} the use of datasets, to document the
  value of datasets in terms of scientific usage, and thus enable data
  providers to perform a rigorous cost-benefit analysis.
\item\label{cmplx} Several of the modeling centres noted that the
  growing intricacy of the experimental specifications -- for
  instance, input data \citep[forcings, see][]{ref:duracketal2017} and
  output variable lists \pipref{juckesetal2015}, as well as the
  crossover requirements between the endorsed MIPs and the DECK
  \pipref{eyringetal2016a}-- made it difficult to verify that
  experiment configurations conform to those specifications. Moreover,
  the protocols inevitably evolve over time, as errors are discovered
  or enhancements proposed, and centres needed to be able to adjust
  their workflows accordingly. The WIP therefore recognized a
  requirement to encode the protocols to be directly ingested by
  workflows, in other words, \emph{machine-readable experiment
    design}. The requirement spans all of the \emph{controlled
    vocabularies} (CVs: for instance the names assigned to models,
  experiments, and output variables) used in the CMIP protocols as
  well as the CMIP6 Data Request \pipref{juckesetal2015}, which must
  be stored in version-controlled, machine-readable formats. Precisely
  documenting the \emph{conformance} of experiments to the protocols
  \pipref{lawrenceetal2012} is an additional requirement.
\item\label{snap} The WIP recognizes that the transition from a
  unitary archive to PCMDI to a federated one has led to many changes
  in the way users interact with the archive. In particular, many
  users of the data (probably a majority) no longer register or
  interact directly with the ESGF. Rather they rely on secondary
  repositories, often replicas of the ``snapshots'' created by users
  (see for instance the \href{https://goo.gl/34AtW6}{IPCC CMIP5 Data
    Factsheet} for a discussion of the snapshots and their coverage).
  This meant that reliance on the ESGF's inventory of registered users
  for any aspect of the infrastructure -- such as tracking usage,
  compliance with licensing requirements, or informing users about
  errata or retractions -- could at best ensure partial coverage of
  the user base. The WIP therefore resolved to a more distributed
  design for several features outline below, which devolve many of
  these features to the datasets themselves rather than the archives.
  One may think of this as a \emph{dataset-centric rather than
    system-centric} design (in software terms, a \emph{pull} rather
  than \emph{push} design): information is made available upon request
  at the user/dataset level, relieving the ESGF implementation of an
  impossible burden.
\end{enumerate}

Based upon these considerations, the WIP has written a detailed set of
recommendations in a series of position papers, available from the
\href{https://www.earthsystemcog.org/projects/wip/}{WIP website}. The
position papers are listed in \appref{wip}, and summarized in the
sections below. The position papers include:

\begin{itemize}
\item \href{https://goo.gl/Z9yHnE}{CDNOT Terms of Reference}: a
  charter for the CMIP6 Data Node Operations Team.
\item \href{https://goo.gl/cMiPE7}{CMIP6 Global Attributes, DRS,
    Filenames, Directory Structure, and CVs}: conventions and
  controlled vocabularies for consistent naming of files and variables.
\item \href{https://goo.gl/dQAEDy}{CMIP6 Persistent Identifiers
    Implementation Plan}: a system of identifying and citing datasets
  used in studies, at a fine grain.
\item \href{https://goo.gl/jqWjQ5}{CMIP6 Replication and Versioning}:
  a system for ensuring reliable and verifiable replication; tracking
  of dataset versions, retractions and errata.
\item \href{https://goo.gl/vKmGM4}{CMIP6 Quality Assurance}: systems
  for ensuring data compliance with rules and conventions listed
  above.
\item \href{https://goo.gl/CZyWq1}{CMIP6 Data Citation and Long Term
    Archival}: a system for generating Document Object Identifies
  (DOIs) to ensure long-term data curation.
\item \href{https://goo.gl/h4HSP1}{CMIP6 Licensing and Access
    Control}: terms of use and licenses to use data.
\item \href{https://goo.gl/Ro97Rv}{CMIP6 ESGF Publication
    Requirements}: linking WIP specifications to the ESGF software
  stack, conventions that software developers can build against.
\item \href{https://goo.gl/qjs8WK}{Errata System for CMIP6}: a system
  for tracking and discovery of reported errata in CMIP6.
\end{itemize}

These position papers have been approved for public distribution. The
WIP continues to develop recommendations, and more will be distributed
on the site. The ones listed here are those available at the time of
writing. These documents are stable for the time being. Should they
evolve, they will be released as new numbered versions.

The recommendations are summarized in the sections below.

\section{The CMIP6 data framework}
\label{sec:dreq}

The CMIP6 data framework has evolved considerably from CMIP5, and
follows the principles of scientific reproducibility (Item~\ref{repro}
in \secref{principles}), and the recognition that the complexity of
the experimental design (Item~\ref{cmplx}) required far greater
degrees of automation and embedding in workflows. This requires the
storage of all elements in the specification in structured text
formats (XML and JSON, for example), and under rigorous version
control. \emph{Machine-readable} specification of as many aspects of
the model output configuration as possible is a WIP design goal.

The data request spans several elements discussed in sub-sections
below.

\subsection{Model inputs}
\label{sec:data-inputs}

Datasets used by the model for configuration of model inputs
\citep[\texttt{input4MIPs}, see][]{ref:duracketal2017} as well as
observations for comparison with models \citep[\texttt{obs4MIPs},
see][]{ref:teixeiraetal2014} are both now organized in the same way as
CMIP model output itself. The datasets follow the versioning
methodologies recommended by the WIP.

\subsection{CMIP6 Data Request}
\label{sec:data-request}

The data request itself \pipref{juckesetal2015} is now available
through the \href{https://goo.gl/iNBQ9m}{DRQ} tool and underlying
database. The DRQ combines definitions of variables and their output
format withr specifications of the objectives they support and the
experiments that they are required for. The entire request is encoded
in an XML database with rigorous type constraints. Important elements
of the request, such as units, cell methods (expressing the sub-grid
processing implicit in the variable definition), and time slices for
required output are defined as controlled vocabularies within the
request to ensure consistency of usage. The request is designed to
enable flexibility, allowing modelling centres to make informed
decisions about the variables they should submit to the CMIP6 archive
from each experiment.


The key elements of a variables definition are the:
\begin{enumerate}
\item specification of the parameter to be calculated in terms of a CF
  standard name and units,
\item an output frequency,
\item a structural specification which includes specification of
  dimensions and of sub-grid processing.
\end{enumerate}

In order to faciliate the cross linking between the 2100 variables
with 196 experiments the request data base allows MIPs to aggregate
variables and experiments into groups. The link between variables and
experiments is then made through the following chain:

\begin{enumerate}
\item A variable group, aggregating variables with priorities specific
  to the MIP defining the group;
\item A request link associates a variable group with an objective and
  a set of request items;
\item Request items which associate a particular time slice with a
  request link and a set of experiments.
\end{enumerate}

This formulation takes into account the complex requests across MIP
boundaries, as when a particular MIP requests that some variable also
be saved from an experiment proposed by a different MIP, or a DECK
experiment.

The data request supports a broad range of users, and has been
provided with a range of different access points.

\begin{enumerate}
\item The XML database provides the reference document;
\item Web pages provide a direct representation of the data base
  content;
\item Excel workbooks provide selected overviews for specific MIPs and
  experiments;
\item A python library provides an interface to the database with some
  build in support functions;
\item A command line tool based on the python library allows quick
  access to simple queries.
\end{enumerate}


The ability to have a machine-readable database of the data request,
with a simple python API, has been an extraordinary resource for the
centres that have taken advantage of it: for instance, being able to
directly interface the request with their own workflows to ensure that
the correct variables were being saved depending on what MIPs they
were planning to run. In addition, it has given us a new-found ability
to simulate and predict the data volume from a future request, a
feature exploited below in \secref{dvol}.

\subsection{Data Reference Syntax}
\label{sec:data-drs}

The organization of the data output follows the Data Reference Syntax
(DRS) first used in CMIP5, and now in somewhat modified form in CMIP6.
The DRS depends on pre-defined vocabularies (\emph{controlled
  vocabularies}, or CVs) for various terms: including the names of
institutions, models, experiments, time frequencies, etc. The CVs
themselves are now a version-controlled set of structured text
documents, and the WIP has taken steps to ensure that there is a
single authoritative source for any CV, on which all elements in the
toolchain will rely. The CVs also appear in file headers themselves,
as global netCDF attributes. These aspects are covered in detail in
the \href{https://goo.gl/cMiPE7}{CMIP6 Global Attributes, DRS,
  Filenames, Directory Structure, and CVs} position paper. A new
element in the DRS is the ability to store both native-grid and
regridded data from the same variable (see discussion below in
\secref{dvol} on the potentially critical role of regridded output).


\subsection{CMIP6 data volumes}
\label{sec:dvol}

As noted, extrapolations based on CMIP3 and CMIP5 lead to some
alarming trends in data volume \citep[see
e.g.,][]{ref:overpecketal2011}. The WIP has undertaken a rigorous
approach to the estimation of future data volumes, rather than simple
extrapolation. Contributions to increase in data volume include the
systematic increase in model resolution and complexity of the
experimental protocol and data request. We consider these separately:

\begin{description}
\item[Resolution] The median horizontal resolution of a CMIP model
  tends to grow with time, and is expected to be 100~km in CMIP6,
  compared to 200~km in CMIP5. The vertical resolution grows in a more
  controlled fashion, at least as far as the data is concerned, as
  these are often requested on a standard set of model levels.
  Similarly the temporal resolution of the data request does not
  increase at the same rate as the model timestep: monthly averages
  remain monthly averages. A doubling of model resolution leads
  therefore to a quadrupling of the data volume, in principle. But
  typically the temporal resolution of the model (though not the data)
  is doubled as well, for reasons of numerical stability. Thus, for an
  $N$-fold increase in horizontal resolution, we require an $N^3$
  increase in computational capacity, which will result in an $N^2$
  increase in data volume. We argue therefore, that data volume $V$ and
  computational capacity $C$ are related as $V \sim C^\frac23$, purely
  from the point of view of resolution. The exponent is even smaller
  if vertical resolution increases are assumed. If we then assume that
  centres will experience an 8-fold increase in $C$ between CMIPs
  (which is optimistic in an era of tight budgets), we can expect a
  4-fold increase in data volume. However, this is not what we
  experienced between CMIP3 and CMIP5. What caused that extraordinary
  50-fold increase in data volume?
\item[Complexity] The answer lies in the complexity of CMIP: the
  complexity of the data request, and of the experimental protocol.
  The data request complexity is related to that of the science: the
  number of processes being studied, and the physical variables
  required for the study. In CPMIP \pipref{balajietal2017}, we have
  attempted a rigorous definition of this sort of complexity, measured
  by the number of physical variables simulated by the model. This, we
  argue, grows not smoothly like resolution, but in very distinct
  generational phase transitions, such as the one from
  atmosphere-ocean models to Earth system models, which involved a
  substantial jump in complexity, the number of physical, chemical,
  and biological species being modeled, as shown in
  \bibref{balajietal2017}.

  The second component of complexity is the experimental protocol.
  With the new structure of CMIP6, with a DECK and 21 endorsed MIPs,
  this would appear to have grown tremendously. We propose as a
  measure of experimental complexity, the \emph{total number of
    simulated years (SYs)} to conform to a given protocol. Note that
  this too is gated by $C$: modeling centres usually make tradeoffs
  between experimental complexity and resolution in deciding their
  level of participation in CMIP6, discussed in detail in
  \bibref{balajietal2017}.
\end{description}

The WIP has recommended two further steps toward ensuring sustainable
growth in data volumes.

\begin{enumerate}
\item The first of these is the consideration of standard horizontal
  resolutions for saving data, as is already done for vertical and
  temporal resolution in the data request. Cross-model analyses
  already cast all data to a common grid in order to evaluate it as an
  ensemble, typically at fairly low resolution. The studies of Knutti
  and colleagues such as \bibref{knuttietal2017} are typically
  performed on the ERA-40 grid ($2^\circ\times 2.5^\circ$) (verify).
  We recommend that for most purposes atmospheric data on the ERA-40
  grid would suffice, with of course exceptions for experiments like
  HighResMIP. A similar recommendation for ocean data (the World Ocean
  Atlas $1^\circ\times 1^\circ$ grid) is made for ocean data, with
  extended discussion of the benefits and losses due to regridding
  \citep[see][]{ref:griffiesetal2014,ref:griffiesetal2016}.
\item The second is the issue of data compression. netCDF4, which is
  the WIP's recommended data format for CMIP6 data, contains the
  ability for compression or deflation \pipref{zivlempel1977}, a
  standard lossless compression technique used in standard tools such
  as \texttt{gzip}, for instance. The amount of compression obtained
  depends upon the ``entropy'' or randomness in the data, with
  smoother data getting more compression.

  Deflation entails computational costs, not only during creation of
  the compressed data, but also every time the data are re-inflated.
  There is also subtle interplay with precision: for instance
  temperatures usually seen in climate models appear to deflate better
  when expressed in Kelvin, rather than Celsius (Paul Durack, personal
  communication), but that is due to the fact that the leading order
  bits are always the same, and thus the data is actually less
  precise. Deflation is also enhanced by reorganizing (``shuffling'')
  the data internally into chunks that have spatial and temporal
  coherence.

  Some in the community have also proposed more aggressive
  \emph{lossy} compression methods \pipref{bakeretal2016}, but the
  WIP, after consideration, believes the loss of precision entailed by
  such methods, and the consequences for scientific results, require
  considerably more evaluation by the community before such methods
  can be accepted as common practice.

  Given all these choices, we undertook a systematic study of the
  behaviour of typical model output files under lossless compression,
  the results are which are \href{https://goo.gl/qkdDnn}{publicly
    available}. The study indicates that standard \texttt{zlib}
  compression in the netCDF4 library with the settings of
  \texttt{deflate=2} (relatively modest, and computationally
  inexpensive), and \texttt{shuffle} (which ensures better
  spatiotemporal homogeneity) ensures the best compromise of results
  between computational cost and data volume savings. Across a coupled
  model, we expect a total savings of about 50\%, with ocean, ice,
  land realms getting the most savings (owing to large fractions of
  masked regions), and atmospheric data the least. This 50\% setting
  has been verified from sample output from some models preparing for
  CMIP6.
\end{enumerate}

The \href{https://goo.gl/iNBQ9m}{DRQ} alluded to above in
\secref{dreq} allows us to make a systematic evolution of these
considerations. The tool allows one to input one's model resolution,
indicate the experiments one intends to participate in, and the data
one intends to save (using DRQ's \emph{priority} attribute). The WIP
carried out a survey of modeling centres in 2016, asking them for
their expected model resolutions, and intentions of participating in
various experiments. We thus have an unprecedented capability now to
simulate data volumes from a future experiment.
\href{https://goo.gl/Ezz5v3}{dreqDataVol.py}, a tool for this purpose
built atop DRQ, is available from the WIP website.

Based on those results, we initially have forecast a data volume of
18~PB for CMIP6. This assumes an overall 50\% compression rate, which
has been approximately verified for at least one CMIP6 model, and
whose compression rates should be quite typical. This number, 18~PB,
is about 6 times the CMIP archive size, and can be explained in terms
of the compounding of modest increases in resolution and complexity,
as explained above. It also explains the dramatic increase between
CMIP3 and CMIP5, which was the result of a massive increase in
complexity. Many models of the CMIP5 era added atmospheric chemistry
and aerosol-cloud feedbacks, sometimes with $\mathcal{O}(100)$
species. CMIP5 was also the one where ESMs and the simulation of the
closing of the carbon cycle made their first appearance. There is no
comparable jump between CMIP5 and CMIP6. CMIP6's innovative
DECK/endorsed-MIP structure should thus be seen as an extension and an
attempt to impose a rational order on CMIP5, rather than a qualitative
leap.

Furthermore, we calculate that the use of the proposed standard model
grids will shrink this volume 10-fold, to 1.8~PB. This is an important
number, as will be seen below in \secref{replica}: the managers of
Tier~1 nodes have indicated that 2~PB is about the practical limit for
replicated storage of combined data from all models. The WIP believes
this target is achievable based on compression and the use of standard
grids. Both of these (the use of netCDF4 compression and regridding)
remain merely recommendations, and the centres are free to choose
whether or not to compress and regrid. But it is likely that data not
so processed will be only available from the originating data node,
and not in the multi-model repositories created by replication as
explained in \secref{replica}. The tracking mechanisms outlined in
\secref{pid} below will allow us to ascertain, after the fact, how
widely used the native grid data may be \emph{vis-\`a-vis} the
regridded subset, and allow us to recalibrate the replicas, as usage
data becomes available. We note also that the providers of the
standard metrics packages like ESMValTool \pipref{eyringetal2016} have
expressed a preference of standard grid data for their analysis, as
regridding from disparate grids adds a lot of complexity to an already
overburdened infrastructure.

\section{Licensing}
\label{sec:licensing}

The WIP's recommended licensing policy is based on an examination of
data usage patterns in CMIP5. First, while the licensing policy called
for registration and acceptance of terms of use, a large fraction,
perhaps a majority of users, actually obtained their data not directly
from ESGF, but from other copies, such as the ``snapshots'' alluded to
above in Item~\ref{snap}, \secref{principles}. The typical usage
pattern, shown in \figref{dark}, involves users making their local
copies, and institutions and user groups making local copies from
ESGF. The WIP \href{https://goo.gl/h4HSP1}{CMIP6 Licensing and Access
  Control} position paper refers to these as ``dark'' repositories and
``dark'' users, invisible to the ESGF system. While this appears to
subvert the licensing and registration policy put in place for CMIP5,
this should not even be seen as a ``bootleg'' process: it is in fact
the most efficient use of limited network bandwidth at the user sites.


\begin{figure*}
  \begin{center}
    \includegraphics[width=175mm]{images/WIP-data-process.png}
  \end{center}
  \caption{Typical data usage pattern in CMIP5 involved users making
    local copies, and user groups making institutional-scale caches
    from ESGF. Figure courtesy Stephan Kindermann, DKRZ, adapted from
    WIP Licensing White Paper.}
  \label{fig:dark}
\end{figure*}

The WIP therefore recommends a licensing policy that inverts this and
removes the impossible task of license enforcement from the
distribution system, and embraces the ``dark'' repositories and users.
To quote the WIP position paper:

\begin{quote}
  The proposal is that (1) a data license be embedded in the data
  files, making it impossible for users to avoid having a copy of the
  license, and (2) the onus on defending the provisions of the license
  be on the original modelling centre...
\end{quote}

The snapshots and emerging resources combining archival and analysis
capabilities such as NCAR's \href{https://goo.gl/sYTxC2}{CMIP
  Analysis Platform} will host data and offload some of the network
provisioning requirements from ESGF nodes themselves.

Modeling centres are offered two choices of \emph{Creative Commons
}licenses: most freely available data will be covered by the
\href{https://goo.gl/CY5m2v}{Creative Commons Attribution ``Share
  Alike'' 4.0 International License}; for centres with more
restrictive policies, the WIP offers the
\href{https://goo.gl/KUNUKq}{Creative Commons Attribution
  ``NonCommercial Share Alike'' 4.0 International License}, which
makes the data freely available for non-commercial use. Further
sharing of the data is allowed, as the license travels with the data.

\section{Citation}
\label{sec:cite}

As noted above in \secref{principles}, the WIP's position on citation
flows from two underlying principles: one, of proper credit and formal
acknowledgment of the authors of datasets; and two, of rigorous
tracking of data provenance. The tracking itself serves the purpose of
scientific reproducibility and traceability, as well as documenting
the utility of datasets.

These principles outlined above also are well-aligned with the
\href{https://goo.gl/Pzb7F6}{Joint Declaration of Data Citation
  Principles} formulated by the Force11 (The Future of Research
Communications and e-Scholarship) Consortium. These amount to an
acknowledgment of the rapid evolution of digital scholarship and
archival, and updating the rules of scholarly publication for the
digital age. We must now recognize data itself, not just the
peer-reviewed literature, as a first-class output of the research
enterprise, and to be curated with the same care as a journal article,
if not more. Most journals and academies also now insist on data used
in the literature to be made publicly available for independent
inquiry and reproduction of results.

Given the complexity of the CMIP6 data request, we expect a total
dataset count of $\mathcal{O}(10^6)$, as shown above in \secref{dvol}.
The WIP therefore recommends two separate approaches to citation, at
different levels of granularity.

The recommendations, detailed in the
\href{https://goo.gl/CZyWq1}{CMIP6 Data Citation and Long Term
  Archival} position paper, recognize two phases to the process: an
initial phase, when the data have been released and preliminary
community analysis is still underway (``Citation of Dynamic Data''),
and a second stage, when the data have been transferred to long-term
archival (LTA) and are now deemed appropriate for interdisciplinary
use, such as in policy studies (``Citation of Stable Data'').


\begin{itemize}
\item The \emph{dynamic data citation} infrastructure is built upon
  the generation of unique \emph{persistent identifiers} (PIDs)
  assigned at the level of an \emph{atomic dataset} (a complete
  timeseries of one variable from one experiment and one model). The
  infrastructure for creating and using PIDs is described below in
  \secref{pid}. New dataset versions (see \secref{version}, below),
  and datasets for the same variable from different models, or
  different experiments, will all receive unique PIDs.

  The WIP strongly recommends that the community document all the
  datasets used in a study using PIDs. However, given that this is
  still preliminary, this does not meet the standards of long-term
  curation, quality, accessibility to be a first-class citable entity.
\item The \emph{stable data citation} infrastructure requires some
  more steps to meet formal requirements for citable data. First, we
  ensure that there has been sufficient community examination of the
  data to qualify as peer review of data quality, Second, there are
  further steps of quality control to ensure standards of
  documentation and completeness, described below in \secref{qa}.

  Once these criteria have been satisfied, a \emph{document object
    identifier} (DOI) will be assigned (``minted'', in the DOI jargon)
  to a dataset. These will be assigned by
  \href{https://www.datacite.org/dois.html}{DataCite}, a DOI-minting
  organization that requires stringent adherence to metadata and
  documentation requirements. For CMIP6, the DDC at DKRZ in Hamburg
  has volunteered to provide the DataCite DOI service.

  Given the formality of the process, a DOI is not assigned to data at
  the same granularity as a PID. The WIP has recommended two
  aggregations suitable for DOI minting:

  \begin{itemize}
  \item \emph{model data}, an aggregation of data produced from a
    single model;
  \item \emph{simulation data}, the aggregation of all the data for a
    certain simulation (a specific experiment in the CMIP6
    experimental hierarchy).
  \end{itemize}

  Once a DataCite DOI is assigned, there is a collective
  responsibility to make its referent \emph{landing page} (see
  \secref{qa}) publicly available in perpetuity. The IPCC-DDCs
  undertake this task of providing long-term access to citable data.

  Stable data are suitable for direct citation in journal articles.
  Data associated with DOIs can also be submitted to the emerging data
  journals, such as GMD's sister publication
  \href{https://www.earth-system-science-data.net/}{Earth System
    Science Data}.
\end{itemize}

\subsection{The PID infrastructure}
\label{sec:pid}

The PID infrastructure is described in the
\href{https://goo.gl/dQAEDy}{CMIP6 Persistent Identifiers
  Implementation Plan} position paper. There are 3 stages:

\begin{itemize}
\item generation of PIDs at the level of individual files, using a
  \emph{handle registry} mechanisms that will generate unique PIDs.
\item generation of PIDs at a coarser granularity of an atomic
  dataset, aggregating all files associated with a single variable from a
  single model running a single experiment, referred to as an
  \emph{atomic dataset} in ESGF terminology.
\end{itemize}

The PIDs assigned at these two levels of the PID hierarchy are unique:
new versions of files or datasets will trigger the creation of new
PIDs, as described in \secref{version}, below. PIDs will also be
generated at higher levels of aggregation, the \emph{model} and
\emph{simulation} data aggregations described above in \secref{cite}.
These levels are dynamic as far as the PID infrastructure is
concerned: new elements (new models, or simulations) can be added to
the aggregation, without modifying the PID. They will become static at
a later date, when the aggregation is deemed stable, and transferred
to long-term archival. This PID architecture is shown in
\figref{pidarch}.


\begin{figure*}
  \begin{center}
    \includegraphics[width=175mm]{images/PID-architecture.png}
  \end{center}
  \caption{PID architecture, showing layers in the PID hierarchy. In
    the lower layers of the hierarchy, PIDs are static once generated,
    and new datasets generate new versions with new PIDs. Figure
    courtesy Tobias Weigel.}
  \label{fig:pidarch}
\end{figure*}

The PID infrastructure is central to the replication
(\secref{replica}) and versioning (\secref{version}) strategies. In
addition, the WIP strongly recommends the publication of
\emph{PIDlists} as supplementary material alongside peer-reviewed
CMIP6 publications, as a mechanism of tracking dataset use and
provenance.

The document further describes methods for generating and registering
PIDs within the system, using the asynchronous messaging system
RabbitMQ. This system, designed in collaboration with ESGF developers,
and shown in \figref{pidflow}, guarantees that PIDs have been
correctly generated in accordance with the guidelines regarding
versioning for instance. This system is thus an extension of the
\texttt{tracking-id} used in CMIP5, but with more rigorous checking
and quality control to ensure that new PIDs are generated when data
are modified.

\begin{figure*}
  \begin{center}
    \includegraphics[width=175mm]{images/PID-workflow.png}
  \end{center}
  \caption{PID workflow, showing the generation and registry of PIDs,
    with checkpoints where compliance is assured. Figure courtesy
    Tobias Weigel.}
  \label{fig:pidflow}
\end{figure*}

\subsection{Quality Assurance}
\label{sec:qa}

The WIP's view of quality assurance (QA) is very broad, and
encompasses the entire data lifecycle, as shown in \figref{qa}. At all
stages, we keep in mind the issues of scientific reproducibility and
provenance capture. Further, as noted in Item~\ref{broad} in
\secref{principles}, the QA procedures must expand the circle of trust
to communities unconnected with the Earth system modeling community
itself.


\begin{figure*}
  \begin{center}
    \includegraphics[width=175mm]{images/WIP-QA.png}
  \end{center}
  \caption{Schematic of the phases of quality assurance, with earlier
    stages in the hands of modeling centres (left), and more formal
    long-term data curation stages at right. Quality assurance is
    applied both to the data (D, above) as well as the metadata (M)
    describing the data. Figure courtesy Martina Stockhause, drawn
    from the WIP's Quality Assurance position paper.}
  \label{fig:qa}
\end{figure*}


QA covers ensuring the scientific content of the data, scientific
descriptions of the data, and consistent practices allowing reliable
use by different elements of the toolchain in \figref{qa}. The early
stages of QA are in the hands of the producers: in fact the cycle of
model development and diagnosis is most elemental QA of all! The
second aspect is ensuring that disseminated data all follow naming
conventions and other metadata, to ensure consistent treatment of data
from different groups and institutions during further stages of the
toolchain. These requirements are directly embedded in the ESGF
publishing process and in tools such as
\href{https://cmor.llnl.gov/}{CMOR} (and its validation component,
\href{https://goo.gl/ApvMJx}{PrePARE}). These
checks (the D1 and M1 phases of QA) ensure that the data conform to
the CMIP6 Data Request, conform to all naming conventions and CVs, and
follow the DRS for storage access patterns. As noted above in
\secref{dreq}, many modeling centres have chosen to embed these steps
directly in their own workflows to ensure conformance as the models
are being run and their output processed.

At this point, as noted in \figref{qa} control is ceded to the ESGF
system, where designated QA nodes perform further QA checks. A
critical new step is the assignment of PIDs (\secref{pid}, the D2
stage of \figref{pidflow}), which, contrary to prior practice, is more
controlled, as the PID is the essential label of datasets across the
data lifecycle.

Beyond this, further stages of QA are to be handled within the ESGF
systems, following procedures outlined in the
\href{https://goo.gl/vKmGM4}{CMIP6 Quality Assurance} position paper.
As described in \secref{cite}, the data are still under scientific
scrutiny by analysts beyond the data producers, in what can be thought
of as a period of community-wide scientific ``quality assurance''.
During this period, modeling centers may correct errors and provide new
versions of datasets. In the final stage, the data pass into long-term
archival, described as the ``bibliometric'' phase in \figref{qa}. Just
prior to LTA, the system will verify minimum standards of
documentation of provenance. This is described in the next section.

\subsection{Documentation of provenance}
\label{sec:doc}

As noted earlier in \secref{dreq}, for data to become a first-class
scientific resource, the methods of its production must be documented
to the fullest extent possible. In particular for CMIP6, this includes
documenting the models and experiments. While of course this is done
through the peer-reviewed literature, we note the need for structured
documentation in machine-readable form for various aspects of search,
discovery and tracking of datasets. 


\begin{figure*}
  \begin{center}
    \includegraphics[width=120mm]{images/ES-DOC-process.png}
  \end{center}
  \caption{Flowchart of ES-DOC documentation process, delineating
    sequence of events and indicating the parties responsible for
    producing the documentation. Figure courtesy Eric Guilyardi and
    Mark Greenslade.}
  \label{fig:esdoc}
\end{figure*}

In CMIP6, the documentation of \emph{models} and \emph{simulations} is
done through the Earth System Documentation
\citep[\href{https://goo.gl/WNwKD9}{ES-DOC},][]{ref:guilyardietal2013}
Project. The various aspects of model documentation are shown in
\figref{esdoc}. The CMIP6 experimental design has been translated into
structured text documents, already available from ES-DOC. ES-DOC has
constructed CVs for the description of the CMIP6 standard model
realms, including a set of short tables (\emph{specialisations}, in
ES-DOC terminology) for each realm. The WIP, and the CMIP Panel, have
strongly recommended that the modeling groups produce the
specialisations as coterminously as possible with the model
construction and documentation in the peer-reviewed literature, to
ensure consistency. ES-DOC provides a variety of user interfaces to
read and write the structured documentation which follows the Common
Information Model of \bibref{lawrenceetal2012}. As models evolve or
differentiate (for example, an Earth system model derived a particular
general circulation model) branches and new versions of the
documentation can be produced in a manner familiar to anyone who works
with version-controlled code.

A critical element in the ESDOC process is the documentation of
\emph{conformances}: steps undertaken by the modeling centres to
ensure that the simulation was conducted as designed. It is here that
we rigorously document which input datasets and versions \citep[e.g the
forcing datasets, see][]{ref:duracketal2017} were used in a
simulation. The conformances will be an important element in the
construction of ensembles in the CMIP6 archive: for instance, we may
wish to subselect only those models that used a particular version of
the forcings. The conformances will continue to grow in importance
following the vision of the DECK providing the continuing thread
anchoring a series of CMIPs \citep[viz. the well-known Figure~1
of][]{ref:eyringetal2016a}. The conformances will be essential in
studies across model generations.

While this method of documentation is unfamiliar to many in the
community, the WIP would like to emphasize its importance, to keep
pace as scientific publication practices evolve in the digital age.
Documentation of software validation \citep[see e.g][]{ref:peng2011}
and structured documentation of complete scientific workflows that can
be independently read and processed, is an increasing trend \citep[see
the special issue on the ``Geoscience Paper of the Future'',
][]{ref:davidetal2016}. We have noted earlier (see Item~\ref{repro} in
\secref{principles}) the currently contentious terrain of scientific
reproducibility, especially important in climate research. Rigorous
documentation remains our best bulwark against challenges to our results.

In keeping with the ``dataset-centric rather than system-centric''
principle of the WIP's approach (Item~\ref{snap} in
\secref{principles}, a user will be able to access the documentation
directly from datasets. This is done in CMIP6 by embedding a global
attribute \texttt{further\_info\_URL} in file headers pointing to the
associated CIM document, which will serve as the landing page for
documentation, from which further exploration (by humans or software)
will take place. The existence and functioning of the landing page is
assured in Stage~M3 of \figref{qa}.

\section{Replication}
\label{sec:replica}

The WIP's replication strategy is covered in the
\href{https://goo.gl/jqWjQ5}{CMIP6 Replication and Versioning}
position paper. The recommendations therein are based on the following
goals:

\begin{itemize}
\item Ensuring at least one copy of a dataset is present at large and
  stable ESGF nodes with a mission of long-term maintenance and
  curation of data;
\item Enabling large-scale data analysis across the federation (see
  Item~\ref{analysis} in \secref{principles});
\item Ensuring continuity of data access even through individual node
  failures;
\item Network load-balancing and performance;
\item Reducing the manual workload related to replication.
\item Building a reliable replication mechanism that can be used not
  only within the federation, but by the secondary repositories
  created by user groups (see discussion in \secref{licensing} around
  \figref{dark}.
\end{itemize}

In conjunction with the ESGF and the International Climate Networking
Working Group (ICNWG), these recommendations have been translated to a
two-pronged strategy.

The basic toolchain for replication are updated versions of the same
software layers used in CMIP5: such as
\href{https://github.com/Prodiguer/synda}{synda} (formerly
\texttt{synchrodata}) and Globus Online \pipref{allenetal2012}, based
on underlying data transport mechanisms such as
\href{https://goo.gl/Z8xcfE}{gridftp} and the older and now deprecated
protocols like \texttt{wget} and \texttt{ftp}.

As before, these layers can be used for \emph{ad hoc} replication by
sites or user groups. For \emph{ad hoc} replication, there is no
obvious mechanism for triggering updates or replication when new data
are published (or retracted, see \secref{version} below). Therefore,
the WIP recommends that designated \emph{replica nodes} maintain a
protocol for automatic replication, shown in \figref{replica}.


\begin{figure*}
  \begin{center}
    \includegraphics[width=120mm]{images/WIP-replication.png}
  \end{center}
  \caption{CMIP6 replication from data nodes to replica centers and
    between replica centers coordinated by a CMIP6 replication team.}
  \label{fig:replica}
\end{figure*}

The replication team will maintain and run a set of monitoring and
notification tools assuring that replicas are up-to-date. The CDNOT is
tasked with ensuring the deployment and function of replica nodes.

A key issue that emerged from the discussions with node managers is
that the replication target has to be of sustainable size. The WIP has
concluded from the discussions that a replication target about 2~PB in
size is the practical (technical and financial) limit for CMIP6.


Given that the projected size of the complete CMIP6 archive is several
times larger (see \secref{dvol}), it is a task for the WIP, in
consultation with the scientific community, to define a
\emph{high-value subset} of the complete CMIP6 archive, to be defined
within the limit of 2~PB. Strategies to achieve this target were
discussed above in \secref{dvol}. We will continue to refine the
approach to replication following some of the directions outlined
there as data begin to arrive.

\section{Versioning}
\label{sec:version}


The WIP position on versioning is based on the principle
(\secref{principles}) of scientific reproducibility. Recognizing that
errors may be found after datasets have been distributed, the WIP
insists that datasets that may have been used downstream be continue
to be publicly available, but marked as superseded. This will allow
users to trace the provenance of published results, even if those
point to retracted data; and further allow the possibility of \emph{a
  posteriori} correction of such results.

The WIP requires a consistent versioning methodology across all the
ESGF data nodes. We note that inconsistent or informal versioning
practices can be invisible to the ESGF infrastructure, which inhibits
any approaches toward traceability across versions: for instance,
yielding files that look like replicas, but with inconsistent data and
checksums.

In close consultation with the ESGF implementation teams, the WIP has
made the following recommendations, described in greater depth in the
\href{https://goo.gl/jqWjQ5}{CMIP6 Replication and Versioning}
position paper. Broadly, they amount to the following:

\begin{itemize}
\item the PID infrastructure of \secref{cite} is the basis of creating
  versions of datasets. PIDs are permanently associated with a
  dataset, and new versions will get a new PID. When new versions are
  published, there will be two-way links created within the PID
  framework, so that one may query a PID for prior or posterior versions.
\item we recommend the unit of versioning be an \emph{atomic dataset}:
  a complete timeseries of one variable from one experiment and one
  model. The implication is that other variables need not be
  republished, if the error is found in a single variable. If an
  entire experiment is retracted and republished, all variables will
  get a consistent version number.
\item the CDNOT will ensure consistent versioning practices at all
  participating data nodes.
\end{itemize}


\subsection{Errata}
\label{sec:errata}


It is worth highlighting in particular the new recommendations
regarding errata. Until CMIP5, we have relied on the ESGF system to
push notifications to registered users regarding retractions and
reported errors. This was found to result in imperfect coverage: as
noted in \secref{licensing}, a substantial fraction of users are
invisible to the ESGF system. Therefore, following the discussion in
\secref{principles} (see Item~\ref{snap}), we have recommended a
design which is dataset-centric rather than system-centric.
Notifications are no longer pushed to users; rather they have a system
where they can query the status of a dataset they are working with. An
\emph{errata client} will allow the user to enter a PID to query its
status; and an \emph{errata server} will return the PIDs associated
with prior or posterior versions of that dataset, if any. Details are
to be found in the \href{https://goo.gl/qjs8WK}{Errata} position
paper.

\conclusions[The future of the global data infrastructure]
\label{sec:summary}

The WIP was formed in response to the explosive growth of CMIP between
CMIP3 and CMIP5, and charged with studying and making recommendations
about the global data infrastructure needed to support CMIP6 and its
future evolution. Our findings reflect the fact that CMIP is no longer
a cottage industry, and a somewhat more formal approach is needed. The
recommendations stop well short of any sort of global governance of
this ``vast machine'', but see many areas where, with a relatively
light touch, we can impose some order and control on the process. We
emphasize here again some of the key aspects of the design:

\begin{itemize}
\item the design is now dataset-centric rather than system-centric:
  see for example the discussion of licensing (\secref{licensing}) and
  dataset tracking (\secref{pid}). This relieves a considerable design
  burden from the ESGF software stack, and further, recognizes that the
  data ecosystem extends well beyond the reach of any software system,
  and data will be used and reused in myriad ways outside anyone's
  control.
\item the standards, conventions, and vocabularies are now stored in
  machine-readable structured text formats like XML and JSON; thus
  enabling software to be built upon them to automate aspects of the
  process. We believe this to be already an urgent need as the
  complexity of experimental protocols is already quite overwhelming,
  which some modeling centres have already begun to take advantage of.
  Moreover, we believe this will also enable and encourage future uses
  and software paths beyond our own narrow imaginations, as
  technologies evolve. Our ability to predict (whether correctly or
  not remains to be seen) the expected CMIP6 data volume is such an
  unexpected outcome.
\item we believe the infrastructure should allow user communities to
  assess the costs of participation as well as the benefits. We
  believe the new PID-based methods of dataset tracking, for instance,
  allow centres to measure which of the data has value downstream. We
  have also noted the importance of citations and fair credit for data
  providers, and attempted to provide a design where data citation is
  both practical and required.
\end{itemize}


Certainly not all issues are resolved, and the validation of some of
our findings will have to await the outcome of CMIP6. Nevertheless, we
believe the discussion in this article provide a sound basis for
beginning to think about the future.

\begin{itemize}
\item There is an increasing blurring of the boundary between weather
  and climate as time and space scales merge \pipref{hoskins2013}.
  This will bring in a new community into our data ecosystems, with
  their own modeling and analysis practices, standards and
  conventions, and other issues. We would recommend a closer
  engagement between these communities in planning the future of
  global data infrastructure.
\item As we have noted, the nature of publication is changing
  \citep[see e.g][]{ref:davidetal2016}. In the future, datasets and
  software with provenance information will be first-class entities of
  scientific publication, alongside the traditional peer-reviewed
  article. In fact it is likely that those will increasingly feature
  in the grey literature and scientific social media: one can imagine
  blog posts and direct annotations on the published literature using
  analysis directly performed on datasets using their PIDs. Data
  analytics at large scale is increasingly moving toward machine
  learning and other directly data-driven methods of analysis, which
  will also be dependent on data with provenance. We believe our
  community needs to pay increasing heed to the status of their data
  and software.
\end{itemize}

The WIP looks to extend its activities as these developments continue.

\appendix

\section{List of WIP position papers}
\label{sec:wip}


\begin{itemize}
\item \href{https://goo.gl/Z9yHnE}{CDNOT Terms of Reference}: a
  charter for the CMIP6 Data Node Operations Team. Authorship: WIP.
\item \href{https://goo.gl/cMiPE7}{CMIP6 Global Attributes, DRS,
    Filenames, Directory Structure, and CVs}: conventions and
  controlled vocabularies for consistent naming of files and
  variables. Authorship: Karl E. Taylor, Martin Juckes, V. Balaji,
  Luca Cinquini, Sébastien Denvil, Paul J. Durack, Mark Elkington,
  Eric Guilyardi, Slava Kharin, Michael Lautenschlager, Bryan
  Lawrence, Denis Nadeau, and Martina Stockhause, and the WIP.
\item \href{https://goo.gl/dQAEDy}{CMIP6 Persistent Identifiers
    Implementation Plan}: a system of identifying and citing datasets
  used in studies, at a fine grain. Authorship: Tobias Weigel, Michael
  Lautenschlager, Martin Juckes and the WIP.
\item \href{https://goo.gl/jqWjQ5}{CMIP6 Replication and Versioning}:
  a system for ensuring reliable and verifiable replication; tracking
  of dataset versions, retractions and errata. Authors: Stephan
  Kindermann, Sebastien Denvil and the WIP.
\item \href{https://goo.gl/vKmGM4}{CMIP6 Quality Assurance}: systems
  for ensuring data compliance with rules and conventions listed
  above. Authorship: Frank Toussaint, Martina Stockhause, Michael
  Lautenschlager and the WIP.
\item \href{https://goo.gl/CZyWq1}{CMIP6 Data Citation and Long Term
    Archival}: a system for generating Document Object Identifies
  (DOIs) to ensure long-term data curation. Authorship: Martina
  Stockhause, Frank Toussaint, Michael Lautenschlager, Bryan Lawrence
  and the WIP.
\item \href{https://goo.gl/h4HSP1}{CMIP6 Licensing and Access
    Control}: terms of use and licenses to use data. Authorship: Bryan
  Lawrence and the WIP.
\item \href{https://goo.gl/Ro97Rv}{CMIP6 ESGF Publication
    Requirements}: linking WIP specifications to the ESGF software
  stack, conventions that software developers can build against.
  Authorship: Martin Juckes and the WIP.
\item \href{https://goo.gl/qjs8WK}{Errata System for CMIP6}: a system
  for tracking and discovery of reported errata in the CMIP6 system.
  Authorship: Guillaume Levavasseur, Sébastien Denvil, Atef Ben
  Nasser, and the WIP.
\end{itemize}

\section{Data and code availability}
\label{sec:code}

\begin{itemize}
\item The software and data used for the study of data compression are
  available at \url{https://goo.gl/qkdDnn}, courtesy Garrett Wright.
\item The software and data used for the prediction of data volumes
  are available at \url{https://goo.gl/Ezz5v3}, courtesy Nalanda
  Sharadjaya.
\end{itemize}

Most of the software referenced here for which the WIP is providing
design guidelines and requirements, but not implementation, including
the ESGF, ESDOC, DRQ software stacks are open source and freely
available. They are autonomous projects and therefore not listed here.

\begin{acknowledgements}
  V. Balaji is supported by the Cooperative Institute for Climate
  Science, Princeton University, Award NA08OAR4320752 from the
  National Oceanic and Atmospheric Administration, U.S. Department of
  Commerce. The statements, findings, conclusions, and recommendations
  are those of the authors and do not necessarily reflect the views of
  Princeton University, the National Oceanic and Atmospheric
  Administration, or the U.S. Department of Commerce. He is grateful
  to the Institut Pierre et Simon Laplace (LABEX-LIPSL) for support in
  2015 during which drafts of this paper were written.

  The research leading to these results has received funding from the
  European Union Seventh Framework program under the IS-ENES2 project
  (grant agreement No. 312979).

  B.N. Lawrence acknowledges additional support from the UK Natural
  Environment Research Council.
\end{acknowledgements}

\bibliographystyle{copernicus}
\bibliography{refs}
\end{document}
